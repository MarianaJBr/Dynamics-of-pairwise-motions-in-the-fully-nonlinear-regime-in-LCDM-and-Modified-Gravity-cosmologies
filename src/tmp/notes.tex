%Model the pairwise velocities of DM particles

%- using HF
%- Exists only for LCDM
% compare against 
%-CLPT 
%-Approximations of the paper
% Test the model against GR and MG simulations
% Probe the validity of the model 
% Scales of validity
% To what accuracy 
% Time evolution

%Explore the amplification of the signal in the v12 ratios (MG vs GR)
               % $Which scales?
               % Which model?
                %When (redshift)?

%Result: We can predict the pairwise velocities of DM particles up to kilo parsec scales with an accuracy of (xx) for LCDM. For beyond GR models, we can predict it with (XY) up to (XX scales) and we found our model a better prediction for the full non linear regime as opposed as others (CLPT). The relevance of looking at this quantity (v12) comes from the fact that even if the clustering ($$\xi$$) shows no impact of the MG model, we find a signal enhanced for (such scale for this model, such other scale for this other model).


%\tableofcontents
%%%%%%%%%%%%%%%%%
    % 1. Motivation from BBKGY hierarchy: Juszkiewicz et al 1999
    %     1. linear and mildly non-linear regimes
    % 2. Extension to the non-linear regime using HaloFit prescription
    % 3. Set of simulations
    %     1. DM, Haloes, Galaxies
    % 4. Results from part 1
    %     1. percentage of accuracy in numerical solution over simulation data
    %         1. great improvement over original JR+99 paper 
    %     2. caveats for haloes and galaxies (biased tracers)
    % 5. Beyond LCDM
    %     1. modified growth functions:
    %         1. scale dependent
    %         2. time dependent
    %     2. Set of simulations
    %         1. DM, Haloes, Galaxies
    %     3. Halo fit for LCDM on top of MG simulations
    %     4. Halomodel for LCDM and extended cosmologies
    %     5. Halomodel: 1-halo term + 2-halo term
    % 6. Results from part 2
    
    % Review works in the literature about pairwise velocities. 

% Goal: To present a model for pairwise velocities of dark matter particles accurate at sub megaparsec  scales up to (X$\%$). 
% We tested our model against LCDM simulations and MG simulations. We discuss the contact of our model with biased tracers. 
%
% Main paper \cite{Juszkiewicz_1999}.
%
% Halos and bullet cluster \cite{2012MNRAS.419.3560T}, and GAMA \cite{2018MNRAS.474.3435L}.

%\MJBold{So far we have only ELEPHANT data. Can we add DUDEG simulations? If we do so, we need to adapt the Halo Model by Gupta et al} 